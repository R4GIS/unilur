% Collection of macros calling the MCQ environments from
%  the exam class instead of the itemize environment

% Backing up item and the itemize environment.
\let\origitem\item
\let\origitemize\itemize
\let\origenditemize\enditemize

% Macro to restore the original itemize environment
\newcommand{\restoreitemize}{
  \let\itemize\origitemize
  \let\enditemize\origenditemize
  \let\item\origitem
}

\newcommand{\opc}[1]{
  \ifstrequal{#1}{on}{
    \let\itemize\oneparchoices
    \let\enditemize\endoneparchoices
    \def\item{\choice}
  }{\restoreitemize}
}

\newcommand{\opcb}[1]{
  \ifstrequal{#1}{on}{
    \let\itemize\oneparcheckboxes
    \let\enditemize\endoneparcheckboxes
    \def\item{\choice}
  }{\restoreitemize}
}

\newcommand{\cb}[1]{
  \ifstrequal{#1}{on}{
    \let\itemize\checkboxes
    \let\enditemize\endcheckboxes
    \def\item{\choice}
  }{\restoreitemize}
}

\newcommand{\ch}[1]{
  \ifstrequal{#1}{on}{
    \let\itemize\choices
    \let\enditemize\endchoices
    \def\item{\choice}
  }{\restoreitemize}
}
\makeatother

% Alternative oneparchoices and oneparcheckboxes
% with equally spaced horizontal items.
% The number of items per line can be adjusted (defaults to 3)
%
% Taken and adapted from the following stackexchange post:
  % http://tex.stackexchange.com/questions/149101/inline-arrangement-using-enumitem

\usepackage{environ, multido, calc}% http://ctan.org/pkg/{environ,multido,calc}
\makeatletter
% Taken from http://tex.stackexchange.com/a/128318/5764
\newcounter{listcount}% Keep track of \item
\newcounter{q}% Enumerate the different items

\NewEnviron{oneparchoicesalt}[1][\perline]{%
  \setcounter{listcount}{0}% Start with 0 \items
  \g@addto@macro{\BODY}{\item\relax\item}% Used to delimit the items; last item identified by \item\relax\item
  \def\item##1\item{% Redefine \item to capture contents
  \def\optarg{##1}%
    \expandafter\ifx\optarg\relax\else% Last item not reached
    \stepcounter{listcount}% Next item being processed
    \expandafter\gdef\csname inlineitem@\thelistcount\endcsname{##1}% Store item in control sequence
    \expandafter\item% Recursively continue processing items
    \fi
    }%
    \BODY% Process environment (save items)
    \setlength{\@tempdima}{\linewidth/#1}% Set length of each \item \parbox; possibly corrent...
        \ifnum\value{listcount}<#1\relax\setlength{\@tempdima}{\linewidth/\value{listcount}}\fi%... if needed
        \par\noindent% Start new non-indented paragraph
      \multido{\i=1+1}{\value{listcount}}{% Insert all items in a \parbox
        \parbox[t]{\@tempdima}{\raggedright\hangindent=1.5em\hangafter=1% Paragraph formatting
        \makebox[1.5em][r]{ \setcounter{q}{\i}
          \opccounter\space}\strut\csname inlineitem@\i\endcsname\strut}\hspace{0pt}}%
    }
    
    
    \newcommand{\perline}{3}
    \newcommand{\opccounter}{\Alph{q})}
    \newcommand{\opcalt}[2][3]{
      % Passing variables with the help of macros
      % and not arguments to homogenize the call.
      \renewcommand{\perline}{#1}
        \renewcommand{\opccounter}{\Alph{q})}
        \ifstrequal{#2}{on}{
          % Switching itemize to MCQ mode
          \let\itemize\oneparchoicesalt
          \let\enditemize\endoneparchoicesalt
        }{\restoreitemize}
      }
      
      \newcommand{\opcbalt}[2][3]{
        % Passing variables with the help of macros
        % and not arguments to homogenize the call.
        \renewcommand{\perline}{#1}
          \renewcommand{\opccounter}{\checkbox@char}
          \ifstrequal{#2}{on}{
            % Switching itemize to MCQ mode
            \let\itemize\oneparchoicesalt
            \let\enditemize\endoneparchoicesalt
          }{\restoreitemize}
}

\makeatother

% Including total page numbers in footer
\lfoot{}
\cfoot{Page \thepage\ of \numpages}
\rfoot{}
